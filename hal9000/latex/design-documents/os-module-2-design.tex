
\chapter{Design of Module \textit{Userprog}}

% ================================================================================= %
\section{Assignment Requirements}

\subsection{Initial Functionality}

Describe in few words (one phrase) what you are starting from in your project. Even if this is something that we all know, it could be a good opportunity for you to be sure you really understand this aspect.

\subsection{Requirements}

Remove the following given official requirements and describe in few words (your own words) what you are requested to do in this part of your project. Even if this is something that we all know, it could be a good opportunity for you to be sure you really understand this aspect. 


The major requirements of the ``Userprog'' assignment, inspired from the original Pintos documentation, are the following:
\begin{itemize}
    \item \textit{System Calls for Process Management}. You have to implement the system calls \textit{SyscallProcessExit()}, \textit{SyscallProcessCreate()}, \textit{SyscallProcessGetPid()}, \textit{SyscallProcessWaitForTermination()} and \textit{SyscallProcessCloseHandle()}.

    \item \textit{System Calls for Thread Management}. You have to implement the system calls \textit{SyscallThreadExit()}, \textit{SyscallThreadCreate()}, \textit{SyscallThreadGetTid()}, \textit{SyscallThreadWaitForTermination()} and \textit{SyscallThreadCloseHandle()}.

    \item \textit{Program Argument Passing}. You have to change new process creation, such that the program arguments to be passed on its user-space stack.
    
    \item \textit{System Calls for File System Access}. You have to implement system calls for opening existing files or creating new files (\textit{SyscallFileCreate()}), reading data from a file \textit{SyscallFileRead()} and closing a file \textit{SyscallFileClose()}).
\end{itemize}

Some additional (and optional) requirements of the ``Userprog'' assignment, specific to UTCN / CS OSD course could be: 
\begin{itemize}
    \item \textit{IPC mechanisms}. You have to add in-kernel support for IPC mechanisms (pipes, shared memory) and the corresponding system calls (also including synchronization mechanisms) for user applications.
    
    \item \textit{Dynamic Memory Allocation Support}. Add in kernel support for mapping new areas in the application's virtual address space and also support managing dynamically allocated memory and corresponding system calls.
    
    \item \textit{Code Sharing Support}. Add support for sharing common code of different processes.
   
%     \item  \textit{Dynamically Linking Libraries}. 
%     \item \textit{Copy-on-Write}
\end{itemize}


The way to allocate requirements on member teams. 
\begin{itemize}
    \item 3-members teams
        \begin{enumerate}
            \item argument passing + validation of system call arguments (pointers)
            
            \item system calls for process management + file system access
            
            \item system calls for thread management
            
        \end{enumerate}

    \item 4-members teams (exceptional cases)
        \begin{enumerate}
            \item argument passing + validation of system call arguments (pointers)
            
            \item system calls for process management + file system access
            
            \item system calls for thread management
            
            \item IPC mechanisms
        \end{enumerate}

     \item optional subjects (for extra points)
        \begin{itemize}
            \item code memory sharing support
            \item dynamic memory allocation support
        \end{itemize}

\end{itemize}


\subsection{Basic Use Cases}

Try to describe a real-life situation, where the requested OS functionality could be useful in a user-application or for a human being. This is also an opportunity for you to better understand what the requirements are and what are they good for. A simple use-case could be enough, if you cannot find more or do not have enough time to describe them.


% ================================================================================= %
\section{Design Description}

\subsection{Needed Data Structures and Functions}

This should be an overview of needed data structure and functions you have to use or add for satisfying the requirements. How the mentioned data structures and functions would be used, must be described in the next subsection ``Detailed Functionality''.


\subsection{Interfaces Between Components}

In this section you must describe the identified interference of your component(s) with the other components (existing or developed by you) in the project. You do not have to get in many details (which go into the next section), but must specify the possible inter-component interactions and specify the existing functions you must use or existing functions you propose for handling such interactions. 


\subsection{Analysis and Detailed Functionality}

Here is where you must describe detailed of your design strategy, like the way the mentioned data structures are used, the way the mentioned functions are implemented and the implied algorithms. 

This must be the main and the most consistent part of your design document.

It very important to have a coherent and clear story (text) here, yet do not forget to put it, when the case in a technical form. So, for instance, when you want to describe an algorithm or the steps a function must take, it would be of real help for your design reader (understand your teacher) to see it as a pseudo-code (see an example below) or at least as an enumerated list. This way, could be easier to see the implied steps and their order, so to better understand your proposed solution.


\subsection{Explanation of Your Design Decisions}

This section is needed, only if you feel extra explanations could be useful in relation to your designed solution. For instance, if you had more alternative, but you chose one of them (which you described in the previous sections), here is where you can explain the reasons of your choice (which could be performance, algorithm complexity, time restrictions or simply your personal preference for the chosen solution). Though, try to keep it short. 

If you had no extra explanation, this section could be omitted at all. 

% ================================================================================= %
\section{Tests}

Your are given in your \OSName{} the tests your solution will be checked against and evaluated and you are not required to develop any addition test. Though, even if the tests are known, it would be helpful for you during the design phase to take a look at the given tests, because that way you can check if your design covers all of them. It would be sufficient for most of tests to just read the comments describing them.

In this section you have to list the tests affecting your design and given a short description of each one (using you own words).


% ================================================================================= %
\section{Observations}

This section is also optional and it is here where you can give your teacher a feedback regarding your design activity.


% ================================================================================= %
\section{Questions that you could be asked}

This section must be removed. This is only to give you some hints for your design. 

Some questions you have to answer (inspired from the original Pintos design templates), but these are not the only possible questions and we insist that your design should not be based exclusively to answering such questions:
\begin{enumerate}
    \item argument passing
        \begin{itemize}
            \item Briefly describe how you implemented argument parsing.  How do you arrange for the elements of argv[] to be in the right order? How do you avoid overflowing the stack page?
            
            \item Why does \OSName{} implement \textit{strtok\_s()} but not \textit{strtok()}?
            
        \end{itemize}

        \item system calls
            \begin{itemize}
                \item Describe how handles are associated with files, processes or threads. Are handles unique within the entire OS or just within a single process?
                
                \item Describe your code for reading and writing user data from the kernel.
                
                \item Suppose a system call causes a full page (4,096 bytes) of data to be copied from user space into the kernel. What is the least and the greatest possible number of inspections of the page table (e.g. calls to \textit{\_VmIsKernelAddress()}) that might result? What about for a system call that only copies 2 bytes of data? Is there room for improvement in these numbers, and how much?
                
                \item Briefly describe your implementation of the ``SyscallProcessWaitForTermination'' system call and how it interacts with process termination.
                
                \item Any access to user program memory at a user-specified address can fail due to a bad pointer value.  Such accesses must cause the system call to fail.  System calls are fraught with such accesses, e.g. a ``SyscallFileRead'' system call requires reading the function's four arguments from the user stack then writing an arbitrary amount of user memory, and any of these can fail at any point.  This poses a design and error-handling problem: how do you best avoid obscuring the primary function of code in a morass of error-handling?  Furthermore, when an error is detected, how do you ensure that all temporarily allocated resources (locks, buffers, etc.) are freed?  In a few paragraphs, describe the strategy or strategies you adopted for managing these issues. Give an example.
                
                
                \item Consider parent process P with child process C. How do you ensure proper synchronization and avoid race conditions when P calls SyscallProcessWaitForTermination(C) before C exits?  After C exits? How do you ensure that all resources are freed in each case? How about when P terminates without waiting, before C exits? After C exits? Are there any special cases?

            \end{itemize}

\end{enumerate}
