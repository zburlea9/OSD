
\chapter{Design of Module \textit{virtualmemory}}

% ================================================================================= %
\section{Assignment Requirements}


\subsection{Initial Functionality}

Describe in few words (one phrase) what you are starting from in your project. Even if this is something that we all know, it could be a good opportunity for you to be sure you really understand this aspect.


\subsection{Requirements}

Remove the following given official requirements and describe in few words (your own words) what you are requested to do in this part of your project. Even if this is something that we all know, it could be a good opportunity for you to be sure you really understand this aspect. 


The requirements of the ``Virtual Memory'' assignment are the following:
\begin{itemize}
    \item \textit{Per Process Quotas}. You have to limit the number of open files a process can have
and the number of physical frames it may use at one time.

    \item \textit{Zero Pages}. You have to implement support for allocating virtual pages filled with
zero bytes.

    \item \textit{System calls}. You have to implement two new system calls for allocating and
de-allocating virtual memory.

		\item \textit{Stack Growth}. You have to add support for dynamically allocated stack.

    \item \textit{Swapping}. You have to implement a page replacement algorithm (i.e. second chance)
and swapping.

		\item \textit{Shared Memory}. You have to add support for multiple processes to share the same memory region.

\end{itemize}


The way to allocate requirements on member teams. 
\begin{itemize}
    \item 3-members teams
        \begin{enumerate}
            \item Per Process Quotas + Zero Pages
            \item System calls + Stack Growth
            \item Swapping
        \end{enumerate}
\end{itemize}

\begin{itemize}
    \item 4-members teams
        \begin{enumerate}
            \item Per Process Quotas + Zero Pages
            \item System calls + Stack Growth
            \item Swapping
            \item Shared Memory
        \end{enumerate}
\end{itemize}

\subsection{Basic Use Cases}

Try to describe a real-life situation, where the requested OS functionality could be useful in a user-application or for a human being. This is also an opportunity for you to better understand what the requirements are and what are they good for. A simple use-case could be enough, if you cannot find more or do not have enough time to describe them.


% ================================================================================= %
\section{Design Description}

\subsection{Needed Data Structures and Functions}

This should be an overview of needed data structure and functions you have to use or add for satisfying the requirements. How the mentioned data structures and functions would be used, must be described in the next subsection ``Detailed Functionality''.

\subsection{Interfaces Between Components}

In this section you must describe the identified interference of your component(s) with the other components (existing or developed by you) in the project. You do not have to get in many details (which go into the next section), but must specify the possible inter-component interactions and specify the existing functions you must use or existing functions you propose for handling such interactions. 


\subsection{Analysis and Detailed Functionality}
Here is where you must describe detailed of your design strategy, like the way the mentioned data structures are used, the way the mentioned functions are implemented and the implied algorithms. 

This must be the main and the most consistent part of your design document.

It very important to have a coherent and clear story (text) here, yet do not forget to put it, when the case in a technical form. So, for instance, when you want to describe an algorithm or the steps a function must take, it would be of real help for your design reader (understand your teacher) to see it as a pseudo-code (see an example below) or at least as an enumerated list. This way, could be easier to see the implied steps and their order, so to better understand your proposed solution.


\subsection{Explanation of Your Design Decisions}

This section is needed, only if you feel extra explanations could be useful in relation to your designed solution. For instance, if you had more alternative, but you chose one of them (which you described in the previous sections), here is where you can explain the reasons of your choice (which could be performance, algorithm complexity, time restrictions or simply your personal preference for the chosen solution). Though, try to keep it short. 

If you had no extra explanation, this section could be omitted at all. 


% ================================================================================= %
\section{Tests}

Your are given in your \OSName{} the tests your solution will be checked against and evaluated and you are not required to develop any addition test. Though, even if the tests are known, it would be helpful for you during the design phase to take a look at the given tests, because that way you can check if your design covers all of them. It would be sufficient for most of tests to just read the comments describing them.

In this section you have to list the tests affecting your design and given a short description of each one (using you own words).


% ================================================================================= %
\section{Observations}

This section is also optional and it is here where you can give your teacher a feedback regarding your design activity.


% ================================================================================= %
\section{Questions that you could be asked}

This section must be removed. This is only to give you some hints for your design. 

Some questions you have to answer (inspired from the original Pintos design templates), but these are not the only possible questions and we insist that your design should not be based exclusively to answering such questions:
\begin{enumerate}
    \item Process Quotas
        \begin{itemize}
            \item When sharing memory between processes the physical frames used should be counted in the quota for each process, how will you achieve this?
        \end{itemize}
				
	  \item Zero Pages
        \begin{itemize}
            \item How will you implement zero pages? How will you support accessing a range of zero memory larger than the swap file? Explain the algorithm required for your implementation.
        \end{itemize}			
				
		\item Stack Growth
        \begin{itemize}
            \item Explain your heuristic for deciding whether a page fault for an invalid virtual address should cause the stack to be extended into the page that faulted. 
        \end{itemize}		

    \item page table management
        \begin{itemize}
            \item In a few paragraphs, describe your code for locating the frame, if any, that contains the data of a given page.
            
            \item How does your code coordinate accessed and dirty bits between kernel and user virtual addresses that alias a single frame, or alternatively how do you avoid the issue?
            
            \item When two user processes both need a new frame at the same time, how are races avoided?
        \end{itemize}
    
    \item page replacement and swapping
        \begin{itemize}
            \item When a frame is required but none is free, some frame must be evicted.  Describe your code for choosing a frame to evict.
            
            \item When a process P obtains a frame that was previously used by a process Q, how do you adjust the page table (and any other data structures) to reflect the frame Q no longer has?
            
            \item Explain the basics of your VM synchronization design. In particular, explain how it prevents deadlock. (Refer to the textbook for an explanation of the necessary conditions for deadlock.)
            
            \item A page fault in process P can cause another process Q's frame to be evicted. How do you ensure that Q cannot access or modify the page during the eviction process?  How do you avoid a race between P evicting Q's frame and Q faulting the page back in?
            
            \item Suppose a page fault in process P causes a page to be read from the file system or swap. How do you ensure that a second process Q cannot interfere by e.g. attempting to evict the frame while it is still being read in?
            
            \item Explain how you handle access to paged-out pages that occur during system calls.  Do you use page faults to bring in pages (as in user programs), or do you have a mechanism for "locking" frames into physical memory, or do you use some other design?  How do you gracefully handle attempted accesses to invalid virtual addresses?

            \item  A single lock for the whole VM system would make synchronization easy, but limit parallelism.  On the other hand, using many locks complicates synchronization and raises the possibility for deadlock but allows for high parallelism.  Explain where your design falls along this continuum and why you chose to design it this way.
        \end{itemize}
				
		\item Shared Memory
        \begin{itemize}
            \item What data do you need to maintain in order to be able to create and access shared memory regions? What data structure will you be using to maintain the keys?
			
						\item After creating a shared memory region, when it is accessed by the second process does the same virtual address need to be returned to the process? Motivate your answer.
			
						\item How will you know when to delete the shared memory region? Take this scenario as an example: a process creates a region and 7 other processes access it, i.e. they call \textit{SyscallVirtualAlloc} with the same key, how do you know when you need to delete the key from your list and implicitly the shared memory region? This region should be valid until the last process which 'opened' it 'closes' it, i.e. calls \textit{SyscallVirtualFree}.
        \end{itemize}

\end{enumerate}
