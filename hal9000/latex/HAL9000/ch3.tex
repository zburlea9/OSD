\chapter{Project 2: Userprog}

\section{Overview}

In the first project you worked with the threading system, interacted a little with the timer
interrupt code and did some inter-processor communication. For the second project you will work on
the user-mode kernel interface system and implement several system calls (syscalls) which will
allow user applications to do useful work.

Currently, \projectname supports loading user-applications without passing any arguments. In this
project you will add support for passing arguments to user applications.

While in the first project you only worked with kernel data, and you didn't have to validate all the
input received in the functions you've written or modified because you trusted the kernel, in this
project you will receive data from user-mode applications which must \textbf{NOT} be trusted.

There MUST be a clear separation between trusted code (which runs in kernel mode) and untrusted code
(running in user-mode). It is not OK for the operating system to crash if a badly written
application wants the OS to read the contents of a file to an invalid address (a NULL pointer for
example).

Put shortly, these are your 3 main objectives for this project: implementing system calls, passing
program arguments to loaded applications and validating all user-input not allowing it to crash the
OS.

The VS configuration used for compiling the code must be \textbf{Userprog}. The only difference
between configuration are the tests which run when building the RunTests project.

\subsection{Userprog Initialization}

While the initialization of the threading system was complicated, initializing support for system
calls is straightforward. The only function involved in this is \func{SyscallCpuInit} which is
called on each CPU.

This function is responsible for setting up a few CPU registers (called MSRs), which define the kernel
entry point of system calls and the RFLAGS (and implicitly the interruptibility) to be used when a 
system call is invoked. For more information you can look-up the SYSCALL and SYSRET instructions in
\cite{intelInstr} and the IA32\_LSTAR, IA32\_FMASK and IA32\_STAR MSRs in \cite{intelSys} Chapter 35
 - Model-Specific Registers (MSRs).

The kernel entry point is \func{SyscallEntry} defined in \file{\_syscall.yasm}. This function is 
responsible for switching to the kernel stack, for saving the user register state, for calling the C
 \func{SyscallHandler} function. In its current implementation the function only retrieves the
syscall number, logs it and sets a STATUS\_NOT\_IMPLEMENTED error code in the user-mode RAX register,
which holds the status of the system call and will be the result seen by the user code. This is the
function where most of your work will be done.

Once control is returned to \func{SyscallEntry}, it will restore the initial register state, restore
the user-mode stack and place the calling instruction pointer in the RCX register, and return to user
land through the SYSRET instruction.

\subsection{Issuing system calls}

The user-mode code, which issues system calls, is also called \func{SyscallEntry} and it is also
defined in a file called \file{\_syscall.yasm}. However, these are separate files and functions.
While the kernel side functions belonged to the HAL9000 project, we can find their user land
counterparts in the UsermodeLibrary project.

This function places all the parameters on the stack, and points the RBP register to point to the
start of the parameters. The system call number is also placed in the R8 register, and the SYSCALL
instruction is executed to perform the switch to kernel mode.

Upon return, the RAX register will hold the status of the system call.

\textbf{NOTE: You don't have to modify these assembly functions or call them directly. When issuing
system calls, you should use the functions defined in the \file{syscall\_if.c}.} For a description
of their parametrization and usage, see \file{syscall\_func.h} - this file is shared by both the
kernel and user applications. These are the functions you will need to implement on the kernel
side.

\subsection{Working with user applications}

\subsubsection{Copying an application to \projectname's filesystem}

All the user-mode applications can be found in the \textit{User-mode -> Applications} VS filter.
However, it is not enough to simply compile an application for it to be seen by \projectname. It
must be copied to its file system. This can be done automatically by running the CopyUmAppsToVm
VS project (\textit{User-mode -> Utils} VS filter).

This project must be built while the HAL9000 VM is powered off, else it won't be able to mount the
file system and copy the application files.

\subsubsection{Loading an application}
\label{sect:AppLoad}

\projectname has a very basic portable executable (PE - \cite{msdnPE}) parser and loader. Each
application written for \projectname must statically link the UsermodeLibrary library file, which 
will provide the application's entry point (both for the main and for secondary threads).

You can see the entry point function for the main thread \func{\_\_start} in \file{um\_lib.c} - this
is the place where the common library and system call interface will be initialized. Once this is
done the actual entry point of the application is called through the \func{\_\_main} function.

Besides the two extra underlines, this is similar to a classic C main function, where the parameters
received are the program arguments (argc and argv).

The entry point for secondary threads is \func{\_\_start\_thread} - this function is implemented in
\file{um\_lib\_helper.c}.

These user-mode application projects must be configured in a special way so \projectname will be
able to load them. We have provided two VS templates, so you can easily add new user applications to
the solution: \file{HAL9000\_UserApplication.zip} and \file{HAL9000\_UserMain.zip}: the first one is
a project template, while the second file is a file template (you should use it when adding the
main.c file).

For VS to recognize these templates, you need to place the project template in \file{My Documents/Visual
Studio 2015/Templates/ProjectTemplates} and the item template in \file{My Documents/Visual
Studio 2015/Templates/ItemTemplates}.

\subsubsection{Virtual Memory Layout}
\label{sect:VirtMemLayout}

The VA range [0xFFFF'8000'0000'0000, 0xFFFF'FFFF'FFFF'FFFF] is mapped in kernel mode, while the
VA range [0x0000'0000'0000'0000', 0x0000'7FFF'FFFF'FFFF] belongs to user-space.

In case you're wondering why there's such a huge gap between the two address spaces, this is due to a
restriction in the x86 CPU architecture, which requires the value of the 47th bit to be reflected in
bits 63-48. Addresses which respect this property are called canonical addresses. The CPU will
generate a \#GP exception when accessing memory using non-canonical addresses.

For more information on the organization of the kernel VA space, you can see the comments above
\func{MmuInitSystem}.

In case of user-mode processes, they are loaded at their preferred base address (you can see this by
going to the Project's properties in VS, then \textit{Linker -> Advanced -> Base Address}. Currently,
all \projectname user-applications have the preferred base address set to 0x1'4000'0000. However, this
is not mandatory - \projectname should be able to load the application anywhere in user-land memory.

The virtual memory managed by the VMM will then start at the image base +
\macro{VA\_ALLOCATIONS\_START\_OFFSET\_FROM\_IMAGE\_BASE}
(currently defined as 1GB). As a result, both the stack and heap allocations will have addresses over
0x1'8000'0000.

\subsubsection{Accessing user memory}

As part of a system call, the kernel must often access memory through pointers provided by a user
program. The kernel must be very careful about doing so, because the user can pass a null pointer,
a pointer to unmapped virtual memory, a pointer to a user address to which it does not have the
proper access rights (e.g: the pointer may be to a read-only area and the system call would write to
that region), or a pointer to kernel virtual address space. All of these types of invalid pointers
must be rejected, without harm to the kernel or other running processes, by failing the system call.

\textbf{NOTE: For this project, we only consider issues which arise from single-thread user
applications. In case of multi-threaded applications, things get more complicated: a valid user-mode
address may become invalid, because it is freed by a different thread than the one which issues the
system call.}

There are at least two reasonable ways to access user memory correctly. The first method is to
verify the validity of a user-provided pointer, then dereference it. If you choose this route, 
you'll want to look at \func{MmuIsBufferValid}. This is the simplest way to handle user memory
access.

The second method is to check only that a user specified pointer doesn't have the 
\textit{VA\_HIGHEST\_VALID\_BIT} bit set (see \func{\_VmIsKernelAddress} and
\func{\_VmIsKernelRange}), then dereference it.
An invalid user pointer will cause a "page fault" that you can handle by modifying the code in
\func{VmmSolvePageFault}. This technique is normally faster, because it takes advantage of the
processor's MMU, so it tends to be used in real kernels.

In either case, you need to make sure not to "leak" resources. For example, suppose that your system
call has acquired a lock, or allocated memory. If you encounter an invalid user pointer afterward,
you must still be sure to release the lock or free the page of memory. If you choose to verify user
pointers before dereferencing them, this should be straightforward. It’s more difficult to handle,
if an invalid pointer causes a page fault, because there’s no way to return an error code from a
memory access.

\textbf{NOTE: A safer, but slower solution would be to map the physical pages described by the user
addresses into kernel space. This way, you would not be bothered if the user application un-maps its
VA to PA translations, as long as it is not able to free the physical pages. Also, this would allow
the SMAP (Supervisor Mode Access Prevention) CPU feature to be activated, causing page faults on
kernel accesses to user memory, thus ensuring the OS doesn't access user-memory by mistake.}

\section{Assignment}

\subsection{Argument Passing}

If you haven't done so yet, you should read the \fullref{sect:AppLoad} now. You will need to work in
\func{\_ThreadSetupMainThreadUserStack} to setup the user-mode stack of the application's main
thread of execution.

You already have the command line and the number of arguments available in the \var{FullCommandLine}
and \var{NumberOfArguments} fields of the PROCESS structure (see \file{process\_internal.h} and
\fullref{sect:Processes} for more details).

All you need to do is to place the arguments on the user stack and return the resulting stack
pointer in the \var{ResultingStack} output parameter. For more information about program loading,
read \fullref{sect:ProgramStart}. We recommend you to use \func{strchr} available in \file{string.h}
for parsing the command arguments.

\subsection{User memory access}

As previously stated, you need to look out for the following violations when accessing
pointers received as parameters in system calls:
\begin{enumerate}
	\item Accessing a NULL pointer.
	\item Accessing an un-mapped user address.
	\item Accessing a user address which does not have enough rights to perform the action required.
Example: if an address is read-only and the system call would put data in the buffer, it should not
do so.
	\item Accessing a kernel-mode virtual address.
\end{enumerate}

\textbf{NOTE: All these conditions must be validated for the whole length of the buffer, and not only
for its start or end position. For example, address 0x4000 and 0x6000 may be valid addresses, but 0x5000
may not.}

All of these conditions are checked in the test suite.

\subsection{System calls}

Implement the system call handler in \func{SyscallHandler}. The only parameter received by this
function is a pointer to the state of the user applications registers at the moment of the system
call. You will find the system call number in the R8 register, a pointer to the parameters in RBP,
and you must place the result of the system call in the RAX register to return the status of the
system call to the user application.

Your goal is to implement all the system calls defined and described in \file{syscall\_func.h}. As
you can see, all the system calls return a STATUS (this is the value of the RAX register).

To implement syscalls, you need to provide ways to read and write data in user virtual
address space.

Also, you may be wondering what's up with the \textit{UM\_HANDLE} data type. The idea is that when
an application opens a file, or creates a new thread or process, it needs a way of later referring to
that created object, i.e. it's useless to open a file if it cannot later be used for reading or
writing data to it. That's where handles come in: each time a user executes a system call, which has
the effect of creating or opening an object (file/thread/process), a \textit{UM\_HANDLE} will be
returned to it.

This handle will later be used by the system calls, which manipulate that class of objects. For files
it would be \func{SyscallFileRead}, \func{SyscallFileWrite} and \func{SyscallFileClose}.

NOTE: While the easy solution might seem to simply return kernel pointers as handles, this is
 not a good design choice, and the implementation isn't so straightforward either.

The reason it is not so straightforward is that in your system call implementation, we require you to
validate all handles. This means, beside validating that the handle is valid and was created for
this process, you should also validate that the handle is of the appropriate type, i.e. you wouldn't
want to be working with a thread handle when reading the contents of a file, or you wouldn't want
process B to be able to access the files opened by process A.

Another reason reason it is a bad design decision is that this effectively introduces an
information disclosure vulnerability into the kernel - 
\href{https://cwe.mitre.org/data/definitions/200.html}{CWE-200 Information Exposure}, i.e. any user-
application can start mapping the kernel environment by having access to information it should
not know.

When you're done with this part and with \fullref{sect:UserExceptions}, \projectname should be
bulletproof. Nothing that a user program can do should ever cause the OS to crash, panic, fail an
assertion, or otherwise malfunction. It is important to emphasize this point: our tests will try to
break your system calls in many, many ways. You need to think of all the corner cases and handle
them.

If a system call is passed an invalid argument, the only acceptable option is to return an error
value.

\subsubsection{File Paths}

\func{SyscallFileCreate} and \func{SyscallProcessCreate} both receive a file path as one of their
parameters. This may be an absolute path, starting from drive letter and specifying the full path
to the file (an example would be "D:\textbackslash Folder\textbackslash File.txt"), or it may be a
relative path, in which case the handling differs for these system calls:
\begin{itemize}
	\item In the case of \func{SyscallFileCreate}, the final path is considered to be relative to the
system drive, i.e. if "Test.ini" is the path given to the syscall, internally it will try to open
"\%SYSTEMDRIVE\%Test.ini". If for example the system drive  is "C:\textbackslash" then the final path will be "C:\textbackslash Test.ini".

	\item In the case of \func{SyscallProcesCreate}, the final path is considered relative to the
applications folder found on the system drive, i.e. if "Apps.exe" is the path given to the syscall,
internally it will try to open "\%SYSTEMDRIVE\%Applications\textbackslash Apps.exe". If for example the system drive is "F:\textbackslash",
then the final path will be "F:\textbackslash Applications \textbackslash Apps.exe".
\end{itemize}

NOTE: You can use \func{IomuGetSystemPartitionPath} to retrieve the system drive.

\subsection{Handling user-mode exceptions}
\label{sect:UserExceptions}

Besides the invalid pointer de-references you could encounter while accessing user memory in your
system call implementation, you also need to protect the OS from the exceptions caused directly by
the running user-code.

User applications are free to use any assembly instructions they want, and as a result, they could try
to access registers which they are not to allowed to access (control registers, MSRs, IO ports and
so on), or these applications may suffer code errors leading to a division by zero exception.

Besides valid page faults caused by the lazy mapping mechanism, regardless of the user exception, you
need to terminate the application. You need to think of a generic way to do this, and you should not
care about the exception which caused the crash.

\section{Source Files}

For this project most of your work will be done in the \textit{usermode} and \textit{executive}
filters. Here's a quick overview of the files you'll be interested in:

\file{thread.c}

You will have to implement parameter passing, so you'll need to modify \func{\_ThreadSetupMainThreadUserStack}
 to properly setup the main thread's stack.

\file{process.h}

\file{process\_defs.h}

These are the public process definitions - these functions may be used by any external components
such as drivers and the syscall interface.

\file{process\_internal.h}

Defines the PROCESS data structure and the functions which should only be called by internal OS
components which manage processes. Some examples include the functions for adding or removing a
thread from a process, and the function to switch the CR3 to the new process’s paging structures.

\file{process.c}

Contains the implementation exposed in all process* headers. For more information see
\fullref{sect:Processes}. You will probably need to implement a per process handle tracking
mechanism here.

\file{um\_application.h}

\file{um\_application.c}

Contains the implementation for reading the user application from disk and loading it into memory.
You will NOT work in these files.

\file{syscall.h}

\file{syscall.c}

Provides the function to initialize the system call dispatching system, \func{SyscallCpuInit}, and
the system call handler: \func{SyscallHandler}. Most of your work will be done in this file.

\file{syscall\_defs.h}

\file{syscall\_func.h}

\file{syscall\_no.h}

These are files which are shared between the kernel and user-mode components. These define the
system call numbers, the system call interface functions, and define basic types used by system
calls such as handles, paging rights and so on.

\file{io.h}

Provides functions to work with devices. When implementing the file system system calls, you will be
interested in \func{IoCreateFile}, \func{IoCloseFile}, \func{IoReadFile} and \func{IoWriteFile}.

\section{FAQ}

\textbf{Q: How much code will I need to write?}

A: Here’s a summary of our reference solution, produced by the hg  diff program. The final row gives
total lines inserted and deleted; a changed line counts as both an insertion and a deletion.

The reference solution represents just one possible solution. Many other solutions are also possible
and many of those differ greatly from the reference solution. Some excellent solutions may not
modify all the files modified by the reference solution, and some may modify files not modified by
the reference solution.

%\begin{figure}
\begin{verbatim}
src/HAL9000/HAL9000.vcxproj             |    5 +
src/HAL9000/HAL9000.vcxproj.filters     |   15 +
src/HAL9000/headers/process_internal.h  |    3 +
src/HAL9000/headers/syscall_struct.h    |   67 +++++++
src/HAL9000/headers/um_handle_manager.h |   46 +++++
src/HAL9000/src/isr.c                   |    9 +-
src/HAL9000/src/process.c               |    8 +
src/HAL9000/src/syscall.c               |  636 ++++++++++++++++++++++++++++++++++++-
src/HAL9000/src/syscall_func.c          |  541 ++++++++++++++++++++++++++++++++++++
src/HAL9000/src/thread.c                |   81 ++++++++-
src/HAL9000/src/um_handle_manager.c     |  248 +++++++++++++++++++++++++++
src/shared/common/syscall_defs.h        |    2 +
src/shared/kernel/heap_tags.h           |    3 +-
13 files changed, 1658 insertions(+), 6 deletions(-)
\end{verbatim}

\newline

\textbf{Q: Any user application run crashes the system with a \#PF exception}

A: The first thing you have to do is make sure the stack is properly set up before any user
application is run. You don't have to implement argument passing from the beginning, however you
should at least reserve space on the stack for the shadow space and the return address.

\newline

\textbf{Q: \projectname asserts with any application}

A: You'll have to implement the \func{SyscallValidateInterface} system call before any system call
is possible. The assert is caused by the fact that the process generated a \#GP in case communication
through system calls is not possible.

\newline

\textbf{Q: Can I set a maximum number of open handles per process?}

A: No.

\newline

\textbf{Q: Do I have to do any changes in user-mode code?}

A: No.

\newline

\textbf{Q: Do I have to implement argument passing for the tests to pass?}

A: No, the argument passing functionality is necessary only for the argument passing tests: "TestUserArgsNone", "TestUserArgsOne", "TestUserArgsMany" and "TestUserArgsAll".